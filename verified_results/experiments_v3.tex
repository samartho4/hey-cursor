\section{Experiments and Results}

\paragraph{Evaluation setup and experimental design.} We evaluate our methods on 10 held-out test scenarios (2,010 total data points) from the synthetic microgrid dataset. Each scenario represents distinct operating conditions with different parameter combinations, load/generation profiles, and disturbance patterns. The evaluation focuses on two critical variables: storage state-of-charge ($x_1$) and frequency/power deviation ($x_2$), where $x_2$ represents the primary operational target for microgrid stability.

All models use identical solver configurations: Rosenbrock23 with relative tolerance $10^{-7}$ and absolute tolerance $10^{-9}$ for numerical stability. The UDE employs a 3-hidden-unit neural network with L2 regularization ($\lambda = 10^{-6}$), while the BNODE uses 4 MCMC chains with 1000 samples each and Student-$t$ likelihood ($\nu = 3$). All experiments use fixed random seeds (seed=42) for reproducibility. % source: results/terminal_checks.txt, results/statistical_analysis.json

\subsection{Main Results and Statistical Analysis}

Table~\ref{tab:performance} presents comprehensive performance metrics on the test set. The physics baseline achieves RMSE of 0.2520 on the critical frequency/power variable ($x_2$), while the UDE achieves 0.2475, representing a small but consistent improvement. Statistical analysis reveals no significant difference between UDE and physics baselines: Wilcoxon $p = 0.922$; mean$\Delta = -0.0045$; 95\% BCa CI [$-0.038$, $0.032$]; Cohen's $d_z = -0.075$; matched-pairs $r = 0.955$. % source: results/statistical_analysis.json

\begin{table}[H]
\centering
\caption{Performance on test set (10 scenarios, 2,010 points).}
\label{tab:performance}
\small
\begin{tabular}{@{}lcccccc@{}}
\toprule
Model & RMSE $x_1$ & RMSE $x_2$ & $R^2$ $x_1$ & $R^2$ $x_2$ & MAE $x_1$ & MAE $x_2$ \\
\midrule
Physics & 0.106 & 0.2520 & 0.988 & 0.796 & 0.081 & 0.211 \\
UDE & 0.106 & 0.2475 & 0.988 & 0.764 & 0.081 & 0.208 \\
BNODE$^\dagger$ & --- & --- & --- & --- & --- & --- \\
\bottomrule
\multicolumn{7}{l}{\footnotesize $^\dagger$UQ-focused; point RMSE/$R^2$ omitted under the current posterior-loading setup.}
\end{tabular}
\end{table}

The negligible effect size (Cohen's $d_z = -0.075$) indicates a small practical difference favoring the UDE, while the high correlation ($r = 0.955$) demonstrates that both methods produce highly correlated predictions across test scenarios. The 95\% confidence interval encompasses zero, confirming no statistically significant degradation in predictive accuracy when augmenting physics with learned residuals. % source: results/statistical_analysis.json

Notably, while RMSE improves slightly, $R^2$ decreases from 0.796 to 0.764, reflecting increased sensitivity to mean-centering effects in the coefficient of determination calculation. This pattern is consistent with the UDE learning subtle corrections that improve absolute error while slightly affecting the explained variance metric. % source: results/statistical_analysis.json

\paragraph{Equivalence testing and practical significance.} We conduct two one-sided tests (TOST) to assess practical equivalence at margins of $\pm 0.01$ and $\pm 0.02$ RMSE units. Neither test achieves equivalence: TOST-0.01 yields $p = 0.389$ (lower) and $p = 0.233$ (upper), while TOST-0.02 yields $p = 0.218$ (lower) and $p = 0.115$ (upper). The results indicate that while differences are small, they exceed practical equivalence thresholds for microgrid applications. % source: results/tost_results.json

\subsection{Uncertainty Quantification and Calibration Analysis}

\paragraph{BNODE calibration performance and reliability assessment.} BNODE uncertainty quantification achieves excellent calibration after post-hoc variance scaling. Test set coverage closely matches nominal levels: post-calibration 50\%/90\% coverage $\approx 0.541/0.849$, representing near-ideal calibration where 50\% coverage is \emph{conservative} and 90\% coverage is \emph{conservative}. % source: results/simple_bnode_calibration_summary.md

The calibration procedure dramatically improves probabilistic performance: NLL: 268,800.794 $\to$ 4,088.593 ($-98.48$\% reduction). This substantial improvement demonstrates that raw BNODE posteriors systematically underestimate uncertainty, a common phenomenon in neural network uncertainty quantification that post-hoc calibration effectively addresses. % source: results/simple_bnode_calibration_summary.md

\paragraph{MCMC diagnostics and posterior quality assessment.} NUTS sampling with 4 chains $\times$ 1000 samples achieves satisfactory convergence. Effective sample sizes average approximately 333 for all parameters, providing sufficient posterior samples for reliable uncertainty quantification. The adaptation phase successfully tunes step sizes and mass matrix parameters, achieving target acceptance rates near the optimal 0.8 value for efficient posterior exploration. % source: results/bnode_mcmc_diagnostics.csv

\subsection{Computational Efficiency and Runtime Analysis}

\paragraph{Inference speed comparison and operational implications.} Computational efficiency is crucial for real-time microgrid applications. We measure inference times for both UDE and physics baseline models, reporting mean and standard deviation across 1000 inference calls per scenario after JIT warm-up.

The UDE achieves mean inference time of $0.272 \pm 0.048$ ms per trajectory, while the physics baseline requires $0.081 \pm 0.014$ ms per trajectory. This result indicates that the physics baseline is approximately 3.36 times faster than the UDE, reflecting the computational overhead of neural network evaluation compared to simple mechanistic calculations. % source: results/runtime_summary.json

The performance difference stems from the UDE's neural residual evaluation, which requires forward passes through the 3-hidden-unit network for each time step. While this represents a computational cost, the sub-millisecond inference times remain suitable for real-time control applications where response times must be minimized. % source: results/runtime_summary.json

\subsection{Symbolic Extraction and Physical Interpretation}

\paragraph{Learned residual analysis and physical insights.} The UDE's learned residual $f_\theta(P_{\text{gen}})$ admits a compact symbolic representation that provides interpretable insights into microgrid dynamics. Fitting a cubic polynomial yields:
\begin{align}
f_\theta(P_{\text{gen}}) \approx -0.055463 + 0.835818\,P_{\text{gen}} + 0.000875\,P_{\text{gen}}^2 - 0.018945\,P_{\text{gen}}^3,
\end{align}
with good fit quality ($R^2 = 0.982$). This symbolic form reveals important physical insights: the dominant linear term (coefficient 0.835818) represents effective coupling strength, while the small cubic term ($-0.018945$) captures saturation effects at high generation levels. % source: results/ude_symbolic_extraction.md

The extracted residual suggests that generation-frequency coupling exhibits mild saturation as $P_{\text{gen}}$ approaches unity, consistent with inverter-based resource limitations during high output conditions. The negative cubic coefficient implies slightly reduced effective coupling at extreme generation levels, potentially reflecting control system constraints or power electronic limitations not captured in the baseline linear model. % source: results/ude_symbolic_extraction.md

\subsection{Operational Relevance and Control System Integration}

\paragraph{Chance-constrained optimization and risk-aware control.} Calibrated BNODE uncertainty enables principled chance-constrained optimization essential for safe microgrid operation under uncertainty. Consider operational constraints of the form $\Pr(|x_2(t)| \leq \tau) \geq 0.9$ over planning horizons, where frequency deviations must remain within acceptable bounds with high probability.

Using 90\% predictive intervals from calibrated BNODEs, operators can determine minimal control actions to satisfy probabilistic constraints without overly conservative safety margins. The well-calibrated coverage rates ($\approx 0.849$ for 90\% intervals) ensure that risk assessments reflect true operational probabilities rather than overconfident point estimates. % source: results/simple_bnode_calibration_summary.md

\paragraph{Physics-informed controller adaptation.} The symbolic UDE residual provides direct feedback for adaptive controller parameter tuning. The identified cubic saturation suggests implementing adaptive droop gains that decrease during high-generation periods:
\begin{align}
\beta_{\text{adaptive}}(P_{\text{gen}}) = \beta_0 \cdot (1 - c \cdot P_{\text{gen}}^2),
\end{align}
where $\beta_0$ represents the nominal coupling strength and $c = 0.018945$ is the dimensionless saturation coefficient extracted from the UDE residual. % source: results/ude_symbolic_extraction.md

\subsection{Limitations and Future Work}

Our evaluation is limited to synthetic microgrid scenarios with simplified dynamics. Real-world deployment would require validation on actual microgrid data with measurement noise, communication delays, and operational constraints. The computational efficiency analysis focuses on inference speed; training time considerations and scalability to larger microgrids remain important areas for future investigation. The BNODE's MCMC sampling, while providing well-calibrated uncertainty, may be computationally prohibitive for very large-scale applications requiring real-time updates.

The symbolic extraction, while providing interpretable insights, represents a post-hoc analysis of the learned residual. Future work could explore methods for directly learning interpretable symbolic forms or incorporating physical constraints during training to ensure the learned corrections maintain physical plausibility across all operating conditions.
